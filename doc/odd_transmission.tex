%%%%%% 
\documentclass[11pt,a4paper,twocolumn,notitlepage]{article}

\usepackage{natbib}
\usepackage{graphicx}
\usepackage{hyperref}

\begin{document}

\title{ODD for the cultural transmission model}
\author{Xavier Rubio-Campillo \and Mark Altaweel \and C. Michael Barton \and Enrico R. Crema}
\maketitle

\section{Purpose}

This model illustrates how theoretical models of social learning strategies \citep{boyd1985,henrich_mcelreath2003,laland2004,mesoudi_2011} can easily be translated into an ABM, by exploring how different modes of transmission combined with different population sizes can affect cultural diversity.


\section{Entities, state variables, and scales}
\subsection{Agents}

Each agent is described by its spatial location $x,y$ and an array composed by $nTraits$ slots that can be one of the values defined by the vector $traitRange$. For example if $nTraits=3$ and $traitRange=\{0,1,2,3,4\}$, an agent can have traits $\{0,2,0\}$ while another one might have $\{0,1,1\}$. This array of numbers represent the cultural traits possessed by each agent. Thus in the example just given, the two agent share the same cultural trait (the number $0$) in the first slot.

\subsection{Environment}

Agents move within a bounded rectangular space sized with dimensions $xDim \times yDim$.

\section{Process overview and scheduling}

The simulation proceeds with a discrete number of time-steps, each where the following two processes update the location and the cultural traits of the agents:

\begin{enumerate}
\item{Movement}
\item{Cultural Transmission}
\end{enumerate}

The order of agents' execution is shuffled every time step, and each phase is simultaneously executed for all agents before moving to the next one.


\section{Design concepts}

\subsection{Basic principles}

The array of cultural traits follows the classical model of Axelrod's dissemination of culture \citep{axelrod_1997}. Transmission mechanisms have been implemented based on the literature cited in the paper.

\subsection{Emergence}

The interaction of the agents follow 4 basic social learning system (vertical, unbiased, prestige-biased and confomist). Each model sees the emergence of a different level of diversity linked to the particular transmission dynamics. 

\subsection{Adaptation}

There is no adaptation.

\subsection{Objectives}

There are no agent objectives.

\subsection{Learning}

There is no learning.

\subsection{Prediction}

There is no prediction.

\subsection{Sensing}

The agent engage into social learning only with individuals located within an euclidean distance $interactionRadius$.

\subsection{Interaction}

Agents copy cultural traits of the other agents following one of the four cultural transmission modes listed above. This leads to different types of populations and traits. 

\subsection{Stochasticity}

The innovation mechanism and the different transmission modes, where cultural genes are changed based on probability, contain stochasticity in the probability of copy/innovate for a particular trait. 

\subsection{Collectives}

Collectives affect agents whereby agents with specific cultural attributes will likely affect other agents that do not have those attributes. %Not sure if I got the first part of this sentence.

\subsection{Observation}

We measure cultural diversity using Simpson's index of diversity $D$:

\begin{equation}
D = 1 - \sum\limits_{i=1}^C{\left(\frac{n_i}{N}\right)^2}
\end{equation}

where $C$ is the total number of possible combination of traits defined by $nTraitRange$ and $nTraits$ (equivalent to the length of the vector $nTraitRange$ elevated by $nTraits$), $n_i$ is the number of agents for the \emph{i}-the combination, and $N$ is the total number of agents ($nAgents$). The index ranges is bounded between 0 (low diversity) and 1 (high diversity). 

\section{Initialization}

A parameter sweep is applied in the four scenarios and using specific parameters varied as referenced in the paper.

\subsection{Agents}

This model is populated by $nAgents$ located at random spatial coordinates. Each of the traits in their cultural vector $traitRange$ is randomly chosen from the values in $traitRange$.

\subsection{Environment}

\section{Input data}

There are text input files in the models' git repository. 
%I don't think input data refers to parameter values ?

\section{Submodels}

\subsection{Movement}

All agents move to a random location within an euclidean distance $moveDistance$ of current position $x,y$.

\subsection{Transmission}
All agents engage into one of the following modes of social learning:

\subsubsection{Vertical Transmission}

With probability $replacementRate$ a random subset of $n$ agents are selected and removed. Then $n$ agents (i.e. the same number of agents being removed) are introduced in the model, each possessing the cultural traits and the spatial coordinates of a randomly selected agent from the previous time-step. However, with probability $innovationRate$ some of these newly added agents will have a new value on one of its cultural traits slots.

\subsubsection{Unbiased Transmission}

Each focal agent first defines its social teacher as a randomly chosen agents located within distance $interactionRadius$. If a social teacher is found, the focal agent choses a random index value from its cultural trait slots, and copies the corresponding value of the social teacher. Thus, for example, if the focal agents have $\{3,2,0\}$,  the social teacher $\{0,1,1\}$, and the random index value is $2$, the updated cultural traits of the focal agent becomes $\{3,1,0\}$. With probability $innovationRate$ the newly acquired is swapped with a random value from $traitRange$.

\subsubsection{Prestige-Biased Transmission}

As in the unbiased transmission model, the focal agents selects a social teacher within distance $interactionRadius$. This time, the probability of being selected as social teacher is however proportional to the trait value at the index number prestigeIndex. More specifically the probability $\pi$ of selecting a social teacher $x$ from a the pool of potential social teachers $P$ (so that all agents in $P$ are located within distance $interactionRadius$ from the focal agent) is given by:

\begin{equation}
 \pi_x = \frac{T_{p,x}+1}{\sum\limits_{i=1}^{P}{(T_{p,i}+1)}} 
\end{equation}

where $T_{p,x}$ and $T_{p,x}$ are respectively the prestige index values of the agent $x$ and $i$. Thus if three agents located within distance $interactionRadius$, have respectively ${3,2,0}$ as trait value at their $prestigeIndex$, the probability for the first agent to be selected is $ \frac{3+1}{(3+1)(2+1)(0+1)} = 0.5$. As for the unbiased model, the actual cultural trait slot being copied is randomly selected, hence portraying social contexts where the learners selects a teacher based on its prestige, but it is not always aware of which cultural trait determines such prestige.  As in the other models, with probability $innovationRate$ the newly acquired is swapped with a random value from $traitRange$.

\subsubsection{Conformist Transmission}

The focal agent defines the pool of social teachers (i.e. all agents located within distance $interactionRadius$) and a randomly selected index value for its cultural trait slots. Then it copies the most common value amongst the social teachers (randomly selecting between the most common ones in case of a tie). With probability $innovationRate$ the newly acquired is swapped with a random value from $traitRange$.

\section{References}
\bibliographystyle{apalike}
\bibliography{references}

\end{document}

