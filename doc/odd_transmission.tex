%%%%%% 
\documentclass[11pt,a4paper,twocolumn,notitlepage]{article}

\usepackage{natbib}
\usepackage{graphicx}
\usepackage{hyperref}

\begin{document}

\title{ODD for the cultural transmission model}
\author{Xavier Rubio-Campillo \and Mark Altaweel \and C. Michael Barton \and Enrico R. Crema}
\maketitle

\section{Purpose}

This model illustrates how theoretical models of social learning strategies can easily be translated into an ABM, by exploring how different modes of transmission combined with different population sizes can affect cultural diversity.

TODO: reference paper?

\section{Entities, state variables, and scales}
\subsection{Agents}

Each agent is described by its spatial location $x,y$ and an array composed by $nTraits$ “slots” that can be one of the values defined by the vector $traitRange$. For example if $nTraits=3$ and $traitRange=\{0,1,2,3,4\}$, an agent can have traits $\{0,2,0\}$ while another one might have $\{0,1,1\}$. This array of numbers represent the cultural traits possessed by each agent. Thus in the example just given, the two agent share the same cultural trait (the number $0$) in the first slot.

\subsection{Environment}

Agents move within a bounded rectangular space sized with dimensions $xDim \times yDim$.

\section{Process overview and scheduling}

The simulation proceeds with a discrete number of time-steps, each where the following two processes update the location and the cultural traits of the agents:

\begin{enumerate}
\item{Movement}
\item{Cultural Transmission}
\end{enumerate}

The order of agents' execution is shuffled every time step, and each phase is simultaneously executed for all agents before moving to the next one.


\section{Design concepts}

\subsection{Basic principles}

The array of cultural traits follows the classical model of Axelrod. Transmission mechanisms have been implemented based on the literature cited in the paper.

\subsection{Emergence}

The interaction of the agents follow 4 basic social learning system (vertical, unbiased, prestige-biased and confomist). Each model sees the emergence of a different level of diversity linked to the particular transmission dynamics. 

\subsection{Adaptation}

No

\subsection{Objectives}

Our agents do not have any objective...

\subsection{Learning}

No

\subsection{Prediction}

No

\subsection{Sensing}

No

\subsection{Interaction}

Agents copy cultural traits of the other agents following one of the four cultural transmission modes. 

\subsection{Stochasticity}

The innovation mechanism and the different transmission modes contain stochasticity in the probability of copy/innovate for a particular trait. 

\subsection{Collectives}

No idea

\subsection{Observation}

We measure the diversity of the system based on Simpson's index of diversity $D$:

$D = 1 - \sum\limits_{c=1}^C{(\frac{t}{T}^2}$

being $C$ the set of possible traits defined by $nTraitRange$, $t$ the number of times a trait $c$ is present in the population of agents, and $T$ the total number of traits.

\section{Initialization}

\subsection{Agents}

This model is populated by $nAgents$ located at random spatial coordinates. Each of the traits in their cultural vector $traitRange$ is randomly chosen from the values in $traitRange$.

\subsection{Environment}

The value of $maxResources$ of each cell is sampled from a uniform distribution $U(0,maxEnergy)$. Current $resources$ of each cell is then copied from its $maxResources$ value.

\section{Input data}

No

\section{Submodels}

\subsection{Movement}

All agents move to a random location within an euclidean distance $moveDistance$ of current position $x,y$.

\subsection{Transmission}
All agents engage into one of the following modes of social learning:

\subsubsection{Vertical Transmission}

With probability $replacementRate$ a random subset of $n$ agents are selected and removed. Then $n$ agents (i.e. the same number of agents being removed) are introduced in the model, each possessing the cultural traits and the spatial coordinates of a randomly selected agent from the previous time-step. However, with probability $innovationRate$ some of these newly added agents will have a new value on one of its cultural traits slots.

\subsubsection{Unbiased Transmission}

Each focal agent first defines its social teacher as a randomly chosen agents located within distance $interactionRadius$. If a social teacher is found, the focal agent choses a random index value from its cultural trait slots, and copies the corresponding value of the social teacher. Thus, for example, if the focal agents have $\{3,2,0\}$,  the social teacher $\{0,1,1\}$, and the random index value is $2$, the updated cultural traits of the focal agent becomes $\{3,1,0\}$. With probability $innovationRate$ the newly acquired is swapped with a random value from $traitRange$.

\subsubsection{Prestige-Biased Transmission}

As in the unbiased transmission model, the focal agents selects a social teacher within distance $interactionRadius$. This time, the probability of being selected as social teacher is however proportional to the trait value at the index number prestigeIndex. More specifically the probability $P$ of selecting a social teacher $x$ from a total population of $N$ agents is given by:

$$ P_x = \frac{T_{p,x}+1}{\sum\limits_{i=1}^{N}{(T_{p,i}+1)}} $$

Thus if three agents located within distance $interactionRadius$, have respectively ${3,2,0}$ as trait value at their $prestigeIndex$, the probability for the first agent to be selected is $ \frac{3+1}{(3+1)(2+1)(0+1)} = 0.5$. As for the unbiased model, the actual cultural trait slot being copied is randomly selected, hence portraying social contexts where the learners selects a teacher based on its prestige, but it is not always aware of which cultural trait determines such prestige.  As in the other models, with probability $innovationRate$ the newly acquired is swapped with a random value from $traitRange$.

\subsubsection{Conformist Transmission}

The focal agent defines the pool of social teachers (i.e. all agents located within distance $interactionRadius$) and a randomly selected index value for its cultural trait slots. Then it copies the most common value amongst the social teachers (randomly selecting between the most common ones in case of a tie). With probability $innovationRate$ the newly acquired is swapped with a random value from $traitRange$.

\end{document}

