%%%%%% 
\documentclass[11pt,a4paper,twocolumn,notitlepage]{article}

\usepackage{natbib}
\usepackage{graphicx}
\usepackage{hyperref}

\begin{document}

\title{ODD for the decision-making model}
\author{Xavier Rubio-Campillo \and Mark Altaweel \and C. Michael Barton \and Enrico R. Crema}
\maketitle

\section{Purpose}

This model illustrates how ABM can be used to explore decision making in the context of a heterogeneous environment. In particular, we will examine how variations in agents' mobility can produce the emergence of a tragedy of the commons.

TODO: reference paper?

\section{Entities, state variables, and scales}
\subsection{Agents}

Each agent is defined by a spatial location $x,y$, its current level of $energy$ and the $energyCost$ it consumes every time step.

\subsection{Environment}

Agents move within a spatial grid with dimensions $xDim \times yDim$. The zone represents a heterogeneous resource landscape where the cell in each set of coordinates $x,y$ is defined by the current level of resources it contains $resources$ and a maximum level $maxResources$. 

\section{Process overview and scheduling}

The simulation proceeds with a discrete number of time-steps, each where the following process updates the population of agents and the resources in the environment.

\begin{enumerate}
\item{Decision making}
\item{Collection}
\item{Cloning}
\item{Energy expenditure}
\item{Resource growth}

The order of agents' execution is shuffled every time step, and each phase is simultaneously executed for all agents before moving to the next one.

\end{enumerate}

\section{Design concepts}

\subsection{Basic principles}

No idea

\subsection{Emergence}

Tragedy of the commons will emerge from the system if all agents have access to the same environmental information (i.e. high $radius$ values).

\subsection{Adaptation}

No

\subsection{Objectives}

The objective of each agent is to collect the highest value of $resources$ per time step.

\subsection{Learning}

No

\subsection{Prediction}

No

\subsection{Sensing}

\subsection{Interaction}

Agents explicitly interact with environment collecting $resources$. Competition between agents is indirect though the environment.

\subsection{Stochasticity}

The only stochastic sources are the order of execution of agents and the choice of cell to move if two (or more) cells have identical $resource$ value.

\subsection{Collectives}

No idea

\subsection{Observation}

Population size is the summary statistic chosen to analyse the outcome of the model.

\section{Initialization}

\subsection{Agents}

This model is initially populated by $nAgents$ located at random spatial coordinates.

\subsection{Environment}

The value of $maxResources$ of each cell is sampled from a uniform distribution $U(0,maxEnergy$. Current $resources$ of each cell is then copied from its $maxResources$ value.

\section{Input data}

No

\section{Submodels}

\subsection{Decision making}

All agents move to a new location based on a greedy decision making process: each agent will move to the cell with highest resources within $radius$ of its current location $x,y$. Chebyshev distance is used to define the list of candidates (i.e. the greatest distance of the two dimensions, $x$ and $y$).

\subsection{Collection}

All agents collect resources up to $maxEnergy$ and the collected amount is removed from cell's $resources$. 

\subsection{Cloning}

Agents whose $energy$ equals $maxEnergy$ produce offspring. The new agents have the same spatial location $x,y$ than their parents, and the $energy$ of both (parent and offspring) is updated to $energy = maxEnergy/2$.

\subsection{Energy expenditure}

Agents decrease their $energy$ level by $energyCost$. If $energy \leq 0$ the agent dies and is removed from the simulation.

\subsection{Resource growth}

The $resources$ of each cell will increase every time step by the parameter $resourceGrowthRate$ up to the local $maxResources$.

\end{document}

